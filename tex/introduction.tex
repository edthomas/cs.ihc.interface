\section{Introdução}

Pizza, da palavra grega $\pi i \tau \tau \alpha$ (\textsl{pitta}, 
``torta, bolo''), foi criada pelos antigos gregos, que cobriam seus pães com 
óleos, ervas e 
queijo \cite{civitello2007cuisine}.  A palavra se espalhou para o
turco como \textsl{pide} e búlgaro, croata e sérvio como \textsl{pita}.
Os romanos desenvolveram a placenta, uma torta feita de uma camada fina de massa
com queijo, mel e folhas de louro.  A pizza moderna surgiu na Itália como
torta Napolitana com tomate.  Em 1889 queijo fora adicionado, quando
Margherita di Savoia, rainha da Itália, em visita à Napoli, foi servida
com uma pizza com as cores da bandeira italiana, vermelho (tomate), 
branco (\textsl{mozzarella}) e verde (manjericão).  Este tipo de pizza é 
chamado de Pizza Margherita desde então.

Estima-se que só em São Paulo, as 6 mil pizzarias vendem 1,5 milhões
de pizzas diariamente \cite{SPPizza}.  No Brasil, variações de pizzas foram
criadas, como a pizza com catupiry onde, por exemplo, a pizza 
quatro queijos
(\textsl{mozzarella}, provolone, parmesão e gorgonzola) torna-se pizza de
cinco queijos, com adicional de catupiry.

É de vital necessidade encontrar o balançeamento perfeito entre quantidade
de massa e recheio para que os consumidores tenham uma experiência melhor
com as pizzas brasileiras.

\subsection{Massa}

A massa da pizza pode variar bastante de acordo com o estilo -- fina em 
uma pizza tipicamente Romana, ou grossa como na pizza de Chicago.  
Tradicionalmente sem temperos, podendo ter adicionado alho, ervas e queijo.
Também pode ser recheada de \textsl{cheddar} e catupiry nas 
bordas.

Nos restaurantes, a pizza é assada em cima de pedra aquecida por 
eletricidade, carvão ou madeira.  Outra opção é a pizza na grelha, onde é
assada diretamente em uma grelha de churrasco.  A pizza grega e de Chicago
é assada em uma forma, sem contato direto com a pedra ou o fogo.

O assamento da pizza autêntica é definido pela \textsl{Associazione Verace 
Pizza Napoletana} (AVPN) \cite{AVPN}.
Após o processo de aquecimento, a massa deve 
ser assada entre 60 -- 90 segundos em uma pedra a $485^oC$ aquecida com fogo
da madeira de carvalho.  Se uma forma for utilizada, esta não pode ter mais de
três milímetros de espressura.

\subsection{Sabores}

Além dos estilos tradicionais (Napolitana, Romana, \textsl{Margherita}, 
\textsl{Marinara}, 
\textsl{Lazio}, \textsl{Capricciosa}, \textsl{Quattro Formaggi}, \textsl{Blanca} 
e \textsl{Alla Casalinga}) as pizzarias atualmente dispõem de cardápios
com várias opções.  Um exemplo é uma pizzaria em Tropea, Calabria, onde
o cardápio possui 500 sabores disponíveis.

Nós sentimos gosto graças à aproximadamente 9 mil receptores gustativos que 
temos espalhados na língua, garganta e no palato mole.  Cada receptor
transporta um grupo de 15 -- 20 células ligadas a fibras nervosas
que levam os impulsos de sabor ao cérebro \cite{Sabor}.

