\section{Discussão}

Sentados em uma mesa, dessas mesas de restaurante italiano, com toalhas 
vermelhas e brancas estilo quase xadrez,
houve muita discussão durante a janta, por parte de que todo italiano
é o mais teimoso possível.
Um dos italianos, Angelo Baigioni, começou uma discussão:
\par - \textsl{Questa non è una pizza autentica italiana. Si deve essere mangiata in 
Italia, con il vino!}

Essa frase gerou uma revolta por parte do dono da pizzaria, Gioberto Migliorini, que retrucou

\par - \textsl{Ma certo che abbiamo la pizza autentica! Sai niente di 
pizza, questa pizza è più autência quella di Napoli. Non vedi che è fatta con 
pomodoro e mozzarella in Italia?} -- logo respondida com o típico provérbio
italiano \textsl{``L'abito non fa il monaco!''} por Argimiro Agnol.

A discussão seguiu
noite a dentro ao som de \textsl{Tarantella Napolitana} com o grupo de 
italianos falando cada vez mais alto, bebendo vinho e discutindo com gestos
tipicamente italianos.

Ao final da janta, a bebida já estava falando por eles. Portanto, conclui-se que
o vinho prefere massas mais finas, de $0,4\,cm$ com recheios de pelo menos
um centímetro de expressura.

Em um próximo estudo, todo o conjunto pode ser adotado como um sistema
bistromático proposto por \cite{Bistromatic},  com o objetivo de calcular a distribuição ideal de ingredientes 
para um determinado grupo
de italianos executando toda a dinâmica italiana, sem que um fator seja 
sobreposto a outro.

