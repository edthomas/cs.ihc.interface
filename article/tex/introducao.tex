\section{Introdução}

O design de IHC visa elaborar um modelo conceitual de entidades e atributos do
domínio e do sistema, estruturar as tarefas e projetar a interação e a interface
de um sistema interativo que apoie os objetivos do usuário. Para isso, podem ser
utilizadas diversas representações, cada qual endereçando uma ou mais questões
de IHC.


	Uma das representações encontradas modela a interação humano-computador
como sendo uma conversa baseada em uma linguagem, a qual denomina-se MoLIC
(Modeling Language for Interaction as Conversation). A MoLIC foi projetada para
apoiar os designers no planejamento da interação com base nas ações dos usuários
visando apoiá-lo em seus objetivos.


	À medida que o design da interação avança, o designer passa a definir a
interface propriamente dita, a qual tornará possível a interação do usuário com
o sistema.



