\section{Introdução}

Pizza, da palavra grega $\pi i \tau \tau \alpha$ (\textsl{pitta}, 
``torta, bolo''), foi criada pelos antigos gregos, que cobriam seus pães com 
óleos, ervas e 
queijo \cite{civitello2007cuisine}.  A palavra se espalhou para o
turco como \textsl{pide} e búlgaro, croata e sérvio como \textsl{pita}.
Os romanos desenvolveram a placenta, uma torta feita de uma camada fina de massa
com queijo, mel e folhas de louro.  A pizza moderna surgiu na Itália como
torta Napolitana com tomate.  Em 1889 queijo fora adicionado, quando
Margherita di Savoia, rainha da Itália, em visita à Napoli, foi servida
com uma pizza com as cores da bandeira italiana, vermelho (tomate), 
branco (\textsl{mozzarella}) e verde (manjericão).  Este tipo de pizza é 
chamado de Pizza Margherita desde então.

Estima-se que só em São Paulo, as 6 mil pizzarias vendem 1,5 milhões
de pizzas diariamente \cite{SPPizza}.  No Brasil, variações de pizzas foram
criadas, como a pizza com catupiry onde, por exemplo, a pizza 
quatro queijos
(\textsl{mozzarella}, provolone, parmesão e gorgonzola) torna-se pizza de
cinco queijos, com adicional de catupiry.

É de vital necessidade encontrar o balançeamento perfeito entre quantidade
de massa e recheio para que os consumidores tenham uma experiência melhor
com as pizzas brasileiras.



